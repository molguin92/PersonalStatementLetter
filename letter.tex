\documentclass[english, 11pt]{kthletter}
\usepackage[T1]{fontenc}
\usepackage[utf8]{inputenc}
\usepackage[backend=bibtex]{biblatex}
\usepackage{csquotes}
\usepackage[inline]{enumitem}

\accounts{}
\address{Division of Information Science \& Engineering\\Malvinas väg 10, 100 44 Stockholm}
\date{Stockholm, \today}
%\location{Malvinas väg 10}
\name{Manuel Olguín Muñoz}
\signature{Manuel Olguín Muñoz\\PhD. Student\\KTH Royal Institute of Technology}
\telephone{+46 73 652 7628}
%\telefax{+46 8 790 7854}
\email{molguin@kth.se}
\web{https://olguin.se}

\begin{document}
\begin{letter}% optional dossier number
  {\textbf{SEC 2019 Student Travel Award Chair}\\Yifan Zhang, PhD.}
\opening{Dear Dr.~Zhang,}

I wish to apply for a Student Travel Grant to attend the 2019 Symposium on Edge Computing (ACM/IEEE SEC'19), to be held in Washington D.C., between November 7\textsuperscript{th} and 9\textsuperscript{th} 2019.

I am a PhD. Student at the KTH Royal Institute of Technology under the supervision of Prof.\ James Gross, beginning the third year of my program.
I obtained my Engineer's Degree in Computer Science from Universidad de Chile in 2017.

My current research pertains to the study and performance evaluations of applications running on Edge Computing infrastructure, in particular with respect to human-in-the-loop applications and control systems. In line with this, last year at SEC'18 in Seattle, WA, I presented a poster titled \emph{Scaling on the Edge - A Benchmarking Suite for Human-in-the-Loop Applications} which, thanks to the invaluable feedback of the attendees, could then be extended to a full paper. Titled \emph{EdgeDroid: An Experimental Approach to Benchmarking Human-in-the-Loop Applications}, this paper was subsequently presented at the 20\textsuperscript{th} Workshop on Mobile Computing Systems and Applications (ACM HotMobile'19). Additionally, for this conference I was awarded two travel grants: the Jubilee Appropriation for Young Researchers from the Knuth \& Alice Wallenberg Foundation in Sweden and a HotMobile Travel Grant. 

I am currently in the middle of a six-month research stay with the Elijah Research Group at Carnegie Mellon University in Pittsburgh, PA, a group led by Prof. Mahadev Satyanarayanan himself. As such, although I will not be presenting a paper at the conference this year, attending SEC'19 would still be a tremendous opportunity for me, given its' prestige as the one of the foremost conferences in the field of Edge Computing. It will allow me to converse with the leading researchers in my field and provide me with fresh ideas for my ongoing research. I excitedly look forward to the presentations of my peers, in particular those about work close to my field such as 
\begin{enumerate*}[label={(\arabic*)}]
    \item \emph{DeFog: Fog Computing Benchmarks}, by J. McChesney \emph{et al.},
    \item \emph{Why Cloud Applications Are not Ready for the Edge (yet).} by C. Nguyen \emph{et al.},
    \item and \emph{Delay-Constrained Offloading of Computationally Intensive Workloads in Edge Computing}, by H. Zeng.
\end{enumerate*}
These presentations will hopefully inspire me with new ideas and offer me the opportunity to establish new connections which could lead to exciting future collaborations.

Finally, attending SEC'19 will allow me to accompany my group colleagues who will be presenting at the venue (\emph{Towards Scalable Edge-Native Applications}, J. Wang \emph{et al.}).

I hope you will consider my application, and I look forward to your response and attending SEC'19.

\closing{Best regards,}

\end{letter}
\end{document}