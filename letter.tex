\documentclass[english, 9pt]{kthletter}
\usepackage[T1]{fontenc}
\usepackage[utf8]{inputenc}
\usepackage[backend=bibtex]{biblatex}
\usepackage{csquotes}


\accounts{}
\address{Malvinas väg 10, 100 44 Stockholm}
\date{Stockholm, \today}
\location{Malvinas väg 10}
\name{Manuel Olguín Muñoz}
\signature{Manuel Olguín Muñoz\\PhD. Student\\KTH Royal Institute of Technology}
\telephone{+46 73 652 7628}
%\telefax{+46 8 790 7854}
\email{molguin@kth.se}
\web{https://olguin.se}

\begin{document}
\begin{letter}% optional dossier number
  {\textbf{HotMobile 2019 Travel Grant Chair}\\Felix Xiaozhu Lin, PhD.}
\opening{Dear Dr.~Lin,}

I wish to apply for a Student Travel Grant to attend the 20\textsuperscript{th} Workshop on Mobile Computing Systems and Applications (ACM HotMobile'19), held in Santa Cruz, CA, on Feb. 27--28 2019.

I am a second year PhD. Student at the KTH Royal Institute of Technology under the supervision of Prof.\ James Gross, doing research on the applications of Edge Computing, in particular with respect to human-in-the-loop applications and control systems.
I obtained my Engineer's Degree in Computer Science from Universidad de Chile in 2017.

At HotMobile'19 I will be presenting a paper titled \emph{EdgeDroid: An Experimental Approach to Benchmarking Human-in-the-Loop Applications}, which represents my work so far in the field of Wearable Cognitive Assistance and Edge Computing.
It was written in close collaboration with the Elijah Research Group at Carnegie Mellon University, led by Prof.\ Mahadev Satyanarayanan.
I have been lucky enough to have been invited to work as a visiting researcher at CMU by \emph{Satya} himself, and I have spent several months there over the last 1.5 years.
Last year, the beginnings of this project were also featured as a poster at the Third ACM/IEEE Symposium on Edge Computing (ACM/IEEE SEC'18).

Presenting at HotMobile'19 is a great opportunity for me, as it will allow me to put forward the novel ideas we have been working on to experts across many fields related to mobile computing.
The work I am presenting is still at an early stage, as is appropriate for the venue, and thus any and all feedback received will be invaluable for the development of the project in the coming future. It will also be the first time I present my work as a speaker at such an important venue, and as such it will surely come to shape the way I view my PhD.\ studies and research in general going forward.

I excitedly look forward to the presentations of my peers, in particular those about work close to my field such as \emph{Enabling Multiple Applications to Simultaneously Augment Reality: Challenges and Directions} by Lebeck et al.\ and \emph{Reconfigurable Streaming for the Mobile Edge} by Tiwari et al.
These presentations will hopefully inspire me with new ideas and offer me the opportunity to establish new connections which could lead to exciting future collaborations.

Finally, as a student based in northern Europe, the expenses for me to attend the workshop are somewhat high and thus any financial assistance will be warmly received.
I hope you will consider my application.

I look forward to your response and attending HotMobile'19.

\closing{Best regards,}

\end{letter}
\end{document}